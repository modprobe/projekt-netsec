\documentclass{scrartcl}
\usepackage[utf8]{inputenc}
\usepackage[T1]{fontenc}
\usepackage[ngerman]{babel}
\usepackage[shortlabels]{enumitem}
\usepackage{listings}
\usepackage{color}
\usepackage{amssymb}

\usepackage[title]{appendix}
\definecolor{dkgreen}{rgb}{0,0.6,0}
\definecolor{gray}{rgb}{0.5,0.5,0.5}
\definecolor{mauve}{rgb}{0.58,0,0.82}

\lstset{frame=tb,
  language=Java,
  aboveskip=3mm,
  belowskip=3mm,
  showstringspaces=false,
  columns=flexible,
  basicstyle={\small\ttfamily},
  numbers=left,
  numberstyle=\tiny\color{gray},
  keywordstyle=\color{blue},
  commentstyle=\color{dkgreen},
  stringstyle=\color{mauve},
  breaklines=true,
  breakatwhitespace=true,
  tabsize=3,
}

\author{Patrick Eickhoff, Alexander Timmermann}
\title{Labreport \#3}
\date{}
\setcounter{section}{0}

\begin{document}
    \maketitle
    \section{HTTP}
    \label{sec:HTTP}

      \subsection{}
      \label{sub:1.1}
        Mit dem Befehl \texttt{telnet www.uni-hamburg.de 80} öffnen wir eine
        Verbindung zu der angegeben Adresse über Port 80 (Http Port).
        Mit der offenen Verbindung erwartet der Host nun unsere Request.
        Nach der üblichen Form für HTTP-Anfragen, fragen wir die
        \textit{home.html} an:
        \begin{lstlisting}
GET /de/inst/ab/svs/home.html HTTP/1.1
Host: www.inf.uni-hamburg.de
        \end{lstlisting}
        Da wir eine HTTP/1.1 Anfrage stellen, müssen wir den Host angegeben,
        da HTTP/1.1 multiple Domains erlaubt.\\\\
        Als Antwort erhalten wir jedoch, dass die gesuchte Seite verschoben
        wurde und nun unter
        \textit{https://www.inf.uni-hamburg.de/de/inst/ab/svs/home.html}
        zu finden ist. Da \texttt{telnet} jedoch keine SSL-Verbindungen
        unterstützt, müssen wir per OpenSSL die HTML anfragen:
        \texttt{openssl s\_client -connect www.inf.uni-hamburg.de:443} (Port 443
        für ssl)
        \begin{lstlisting}
 GET /de/inst/ab/svs/home.html
 Host: www.inf.uni-hamburg.de
        \end{lstlisting}
        Im Kopf der HTML können wir sehen, dass
        \textit{/assets/application-11e3b49e605ff8ba1f01d2\\
                75bd36850edfdfc1fbbb8c22e55fae1baf643a00d0.css}
        der Stylesheet ist, den wir suchen. Da SSL jedoch eine sichere Verbindung
        ist, haben wir kaum Zeit unsere nächste Anfrage zu stellen:
        \begin{lstlisting}
 GET /assets/application-11e3b49e605ff8ba1f01d275bd36850edfdfc1fbb
 b8c22e55fae1baf643a00d0.css
 Host: www.inf.uni-hamburg.de
        \end{lstlisting}
  \section{SMTP (Mail-Spoofing)}
  \label{sec:SMTP (Mail-Spoofing)}
    \subsection{}
    \label{sub:2.1}
      Mittels \texttt{netcat mailhost.informatik.uni-hamburg.de 25} verbinden
      wir uns mit dem SMTP-Server des Informatikums.
      \begin{lstlisting}
EHLO fake.doma.in
MAIL FROM: <definitelynotfake@informatik.uni-hamburg.de>
RCPT TO: <0pfer@informatik.uni-hamburg.de>
DATA
From: Serious Guy <definitelynotfake@informatik.uni-hamburg.de>
To: P. Fer <0pfer@informatik.uni-hamburg.de>
Date: Mon, 10 Apr 2016 10:00:00 -0400
Subject: PLEASE ENTER PIN NOW

Ein sprechender Elch will meine Kreditkartennummer! Das find ich fair.
https://www.youtube.com/watch?v=IfXMN3VhikA
.
QUIT
      \end{lstlisting}
      Wenn man nun den Quelltext unserer Fake-Mail und einer normalen Mail
      vergleicht sieht man einige Unterschiede:\\
      Zum einem ist die Fake-Mail nicht im MIME-Format, wie normalerweise
      üblich. Sehr gut lässt sich auch erkennen, dass Nachrichten von
      authentifizierten Usern des RZ auch als solche im Quelltext sichtbar
      sind: \texttt{(Authenticated sender: 123Mustermann)}. Dies sind nur
      einige der Unterschiede zwischen einer echten und unserer gefälschten
      Mail.

  \section{DNS-Spoofing}
  \label{sec:DNS-Spoofing}
    \subsection{}
    \label{sub:3.1}
      Nach einiger Interaktion mit dem Lizenzserver, fällt uns auf, dass keine
      Authentifikation zwischen Klient und Server gefordert wird. Ausserdem ist
      die Bestätigung eienr Lizenz vom Server zum Klienten nur der String
      \texttt{SERIAL\_VALID=1}. Dies lässt sich leicht fälschen, wenn wir
      einfach unseren eigenen Server mittels DNS-Spoofing als Lizenzserver
      ausgeben.
    \subsection{}
    \label{sub:3.2}
      Um den Lizenclient auszutricksen, müssen wir zuerst sicherstellen, dass
      er sich mit unserem eigenem Server anstatt dem Lizenzserver verbindet.
      Obwohl die \textit{LicenseClient.class} nicht einfach auslesbar ist,
      können wir mittels \texttt{strings LicenseClient.class} herausfinden,
      dass der Klient immer mit der selben Hostadresse \textit{licenseserver}
      verbindet.
      Nun müssen wir nur noch in der \textit{hosts}-Datei folgenden Eintrag
      hinzufügen: \textit{127.0.1.2 licenseserver}, wobei 127.0.1.2 die
      IP-Adresse ist, auf der wir unseren eigenen Server laufen lassen.\\\\
      Unseren Server haben wir in \textit{Ruby} geschrieben\\
      (Source:http://www.tutorialspoint.com/ruby/ruby\_socket\_programming.htm,
      sh. Appendix A). Wenn der Server angesprochen wird, tut dieser nichts
      anderes, als irgendeine Eingabe vom Klienten zu nehmen und mit
      \texttt{SERIAL\_VALID=1} zu antworten.
    \subsection{}
    \label{sub:3.3}
      Um sich vor DNS-Spoofing zu schützen hat der Betreiber mehrere
      Möglichkkeiten: Zum einem können so simple DNS-Angriffe verhindert werden,
      wenn statt der \textit{/etc/hosts} Datei direkt eine DNS-Query verwendet
      wird. Dies ist aber auch nur begrenzter Schutz, da auch die DNS-Resolver
      mit sog. \textit{cache poisoning} manipuiliert werden können. Einen
      besseren Schutz bieten Methoden wie: \\
      \begin{enumerate}
        \item Zufällig Groß- und Kleinbuchstaben verwenden, da diese nicht im
              Cache des Resolvers stehen, aber von Name-Servern beim auflösen
              der IP-Adresse ignoriert werden.
        \item Die Query-ID zufällig setzen, sodass Queries nicht gezielt
              manipuliert werden können.
      \end{enumerate}
      Am einfachsten wäre es vermutlich innerhalb der LicenseClient-Datei den
      Namen mit Groß- und Kleinschreibung zu randomisieren.\\
      Für weitere Informationen siehe Quelle:
      \textit{http://www.esecurityplanet.com/network-security/how-to-prevent-dns-attacks.html}

  \section{License-Server(Bruteforce-Angriff)}
  \label{sec:License-Server(Bruteforce-Angriff)}
    \subsection{}
    \label{sub:4.1}
      Den Bruteforce-Angriff haben wir als Python-Skript geschrieben (sh.
      Appendix B) basierend auf dem Passwort-Bruteforce. Als problematisch
      erwies sich jedoch, dass der LicenseServer bei zu vielen Anfragen bzw.
      Versuchen die Verbindung schließt und einen Neuaufbau für ca. 10s
      ablehnt. Da immer wieder 10s zu warten die Rechenzeit enorm erhöhen
      würde, führen wir das Testen der einzelnen Lizenzschlüssel nebenläufig
      durch. Dies ist deutlich effizienter.
    \subsection{}
    \label{sub:4.2}
      Eine Möglichkeit für den Betreiber sich gegen Bruteforceangriffe zu
      schützen, ist einen größeren Zeitabstand zwischen Anfragen zu fordern.
      Auf diese Weise würde ein Bruteforce-Angriff einen enormen Zeitaufwand
      haben. Eine weitere Möglichkeit ist, bei hoher Anzahl subsequenter
      Anfragen von der selben IP-Adresse, diese zu sperren (Blacklisting).
    \subsection{}
    \label{sub:4.3}
      Seriennummern werden nach folgender Formel erstellt:
      \[
        s = (n \cdot 3133700)~mod~10^8, n \in \mathbb{N}
      \]
      Zur Überprüfung reicht es folglich beim Server, die empfangene Seriennummer
      auf eine restfreie Division durch $3133700$ und die Länge von 8 Zeichen
      zu prüfen, also z.B.:
      \begin{lstlisting}[language=ruby]
def validate_serial(input)
    return (input.to_i % 3133700 == 0) && (input.length == 8)
end
      \end{lstlisting}

  \section{Iplementieren eines TCP-Chats}
  \label{sec:Iplementieren eines TCP-Chats}
    \subsection{}
    \label{sub:5.1}
      Mithilfe unserer \textit{UDPReceiver.java} (siehe Appendix C) können wir
      die einzelnen UDP-Packete empfangen und auf der Konsole ausgeben.
      Da die Packete jedoch nur Teile der beiden URL's enthalten, müssen wir
      die einzelnen Packete selbst zusammen puzzeln. Die URL's lauten:
      \textit{http://www.oracle.com/technetwork/java/socket-140484.html} und\\
      \textit{https://code.google.com/archive/p/example-of-servlet/}. Die
      2. URL wurde anders übermittelt, da jedoch der Project-Hosting-Service
      von Google Anfang 2016 eingestellt wurde, findet man dies nun im Archiv.
    \subsection{}
    \label{sub:5.2}
      Nach Zusammenfassen und etwas umschreiben des Tutorials von Oracle, läuft
      unser Server dann über \textit{localhost} auf dem Port 4444. Wenn wir
      diesen jetzt über \texttt{telnet localhost 4444} ansprechen, können
      wir per Konsole Daten an den Server übertragen. Ohne das wir die
      Funktionalität des Servers geändert haben, gibt er dem Klienten jedoch
      nur seine Eingabe als Ausgabe zurück.
    \subsection{}
    \label{sub:5.3}
      Die ursprüngliche Implementation von Server und Klient, reicht nicht aus,
      um einen tatsächlichen Chatraum mehrerer User zu verwirklichen.
      Jeder der User, deren Verbindung jeweils über einen eigenen Thread
      verwaltet wird, muss nun mit den anderen Usern kommunizieren können,
      um Nachrichten auszutauschen. Da Kommunikation zwischen einzelnen
      unabhängigen Threads nicht einfach zu realisieren ist, soll der Server
      die Kommunikation verwallten.\\
      Hierzu haben wir ein Beobachtermuster implementiert, sodass der Server
      alle Threads in einer Liste speichert und die Threads dem Server mitteilen
      , wenn sie eine Nachricht senden. In der dafür vorgesehen \texttt{notify}-
      Methode, wird dann an alle registrierten Threads der \texttt{print}-Befehl
      weitergereicht.
      \begin{lstlisting}[language=java]
public void notify(String message, int ID, String User) {
	for (ClientThread t : connections) {
		if (t.ID != ID) {
			t.print(User + ":" + message);
		}
	}
}
      \end{lstlisting}
      Ein Thread, der seine Client-Verbindung schließt, wird aus der Liste
      \textit{connections} entfernt.
    \subsection{}
    \label{sub:5.4}
      Die \textit{Useradmin.java} aus Aufgabenblatt2 liefert uns eine einfache
      Möglichkeit eine Authentifikation zu implementieren. Nachdem wir die
      Klasse importiert haben, können wir auf die \texttt{checkUser} Methode
      zugreifen, um zu prüfen, ob der User in unserer Datenbank vorliegt und
      das eingegebene Passwort korrekt ist.\\
      Bei Verbindungsaufbau eines Klienten, sendet der Server eine Anfrage nach
      Username und Passwort. Damit wir das Passwort nicht als String im
      Speicher stehen haben, lesen wir den Inputstream Byte für Byte aus und
      konvertieren diese zu Charactern, die wir einer Liste hinzufügen, bis wir
      das Terminierungszeichen \texttt{\\r} lesen. Diese Liste lässt sich dann
      in ein char-Array umwandeln.
      \begin{lstlisting}[language=java]
BufferedReader in = null;
try {
	PrintWriter out = new PrintWriter(
                    client.getOutputStream(), true);
	in = new BufferedReader(new InputStreamReader(
			client.getInputStream()));
	out.println("Username: ");
	String username = in.readLine();
	out.println("Password:");
	List<Character> l = new ArrayList<Character>();
	do {
		char c = (char) in.read();
		if (c == '\r') {
			break;
		}
		l.add(c);
	} while (true);
	char[] password = new char[l.size()];
	int i = 0;
	for (Character c : l) {
		password[i] = c.charValue();
		i++;
	}
	l = null;
 \end{lstlisting}
      Der Server prüft intern, ob die User-Passwort-Kombination gültig ist.
      Ist sie es nicht, wird sofort die Verbindung geschlossen. Wenn der User
      sich erfolgreich authentifiziert, trägt der Server den User in die Liste
      eingeloggter User ein. So wird verhindert, dass sich verschiedene Personen
      mit dem selben Benutzer anmelden. Beim Trennen der Verbindung wird der
      User wieder aus der Liste entfernt.
      \newpage

      \begin{appendices}
        \section{DNS-Spoofing Ruby}
          \lstinputlisting[language=ruby]{tcp_server_spoof.rb}
         \section{License-Server-Bruteforce}
          \lstinputlisting[language=python]{serialbruteforce.py}
        \section{UDP-Receiver}
          \lstinputlisting{UDPReceiver.java}
        \section{TCP-Chat}
          \lstinputlisting[caption=ChatServer.java]{tcp_chat/tcp_chat/ChatServer.java}
          \lstinputlisting[caption=ClientThread.java]{tcp_chat/tcp_chat/ClientThread.java}
      \end{appendices}

\end{document}
