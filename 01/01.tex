\documentclass{scrartcl}
\usepackage[utf8]{inputenc}
\usepackage[T1]{fontenc}
\usepackage[ngerman]{babel}
\usepackage[shortlabels]{enumitem}

\author{Patrick Eickhoff, Alexander Timmermann}
\title{Labreport \#1}
\setcounter{section}{1}

\begin{document}
    \maketitle
    \section*{Aufgabe 1}
    \label{sec:Aufgabe 1}

    \subsection{Arbeiten mit der Shell}
    \label{sub:Arbeiten mit der Shell}

    \subsubsection{man ls}
    \label{subs:man ls}
        Mit dem Befehl \texttt{man ls} kann man sich die sog. \textit{man page}
        des ls-Befehls ansehen. Dort werden alle Funktionen des Programmes erläutert
        und ausführlich dokumentiert.

    \subsubsection{ls \--\--help}
    \label{subs:ls --help}
        Mit dem Befehl \texttt{ls \--\--help} kann man sich eine Kurzreferenz des
        ls-Befehls anzeigen lassen. Dort werden die wichtigsten Informationen
        zusammengefasst und auf Hintergrundinfos verzichtet.

    \subsubsection{script}
    \label{subs:script}
        Der Befehl \texttt{script} kann ein Transkript einer Shell-Session
        speichern. Insbesondere für das Schreiben des Lab Reports ist dies
        als Notiz sehr hilfreich.

    \subsection{Benutzerkonten und -verwaltung}
    \label{sub:Benutzerkonten und -verwaltung}

    \begin{enumerate}[1.]
        \item Mit dem Befehl \texttt{adduser labmate} legen wir den User
              \textit{labmate} an. Mit einer interaktiven Abfrage wird das
              Passwort des Users gesetzt.
        \item Mit den Befehlen \texttt{groups} oder \texttt{id} lassen sich die
              Gruppen des Users anzeigen. Zu Beginn befindet sich der User
              lediglich in einer automatisch erstellten Gruppe, die nach dem
              Username benannt ist.
        \item Mit dem Befehl \texttt{addgroup labortests} legen wir die Gruppe
              \textit{labortests} an.
        \item Mit dem Befehl \texttt{usermod -a -G labortests labmate} fügen wir
              den Benutzer \textit{labmate} zur Gruppe \textit{labortests} hinzu.
        \item Damit der neue Benutzer den \textit{sudo}-Befehl benutzen darf,
              muss er zu einer Gruppe hinzugefügt werden, die in der Datei
              \textit{/etc/sudoers} konfiguriert ist. Auf dem Test-System ist
              dies \textit{admin}. Dies bewerkstelligen wir mit dem Befehl
              \texttt{usermod -a -G admin labmate}.
    \end{enumerate}

    \subsection{Datei- und Rechteverwaltung}
    \label{sub:Datei- und Rechteverwaltung}

    \begin{enumerate}[1.]
      \item Mit dem Befehl \texttt{sudo -su labmate} öffnen wir eine neue \textit{shell}
            als Nutzer \textit{labmate}.
      \item Mit dem Befel \texttt{cd ~} wechseln wir in das home-Verzeichnis des
            aktuellen Nutzers. Der Befehl \texttt{pwd} zeigt den aktuellen
            Verzeichnispfad \textit{/home/labmate} an.
      \item Mit dem Befehl \texttt{mkdir labreports} legen wir das Verzeichnis
            \textit{/lapreports} an.
      \item [4/5/6.] Nachdem wir die Datei \textit{bericht1.txt} mit den Befehlen
            \texttt{touch bericht1.txt \&\& pico bericht1.txt} bearbeitet und
            gespeichert haben, setzen wir die Gruppe dieser Datei mittels
            \texttt{chgrp labortests bericht01.txt} auf \textit{labortests}.
            Im Anschluss setzen wir mittels \texttt{chmod 0660} die Rechte für
            Besitzer, sowie Teilnehmer der Gruppe auf \textit{rw} (read\&write).
            Dieser Befehl setzt sich aus 4 Oktalzahlen zusammen: Das Sticky-Bit,
            der Owner, die Group und Others. Die Oktalzahl für die Rechte setzt sich
            dann aus den 3-Bit für rwx zusammen.
      \item [8.] Nachdem wir mittels \texttt{chmod 0660 labortests} die Rechte für
            dieses Verzeichnis verändert haben, versuchen wir mit dem Befehl
            \texttt{cd labortests} in das Verzeichnis zu wechseln. Dies ist jedoch
            nicht möglich, da wir keine execute-Rechte für das Verzeichnis besitzen.
      \item [9.] Das selbe Verhalten beobachten wir, wenn wir versuchen in das Verzeichnis
            \textit{/root} zu wechseln, da wir kein root-user sind.
      \item [10.] Zuerst wechseln wir in das Verzeichnis \textit{/opt} und konfigurieren
            mit folgenden Befehlen das neue Verzeichnis:
            \texttt{mkdir test \&\& chown labmate test \&\& chgrp user labmate}.
            Mit dem Befehl \texttt{chmod 0770 test} geben wir, wie oben beschrieben,
            dem Besitzer, sowie der Gruppe von \textit{test} Lese-,Schreib- und
            Ausführungsrechte (rwx).
      \item [11.] Mit dem Befehl \texttt{cp /home/labmate/labreports/index.html /opt/test}
            kopieren wir die Datei \textit{index.html} in das Verzeichnis \textit{/opt/test}.
      \item [12.] Zuerst setzen wir die Gruppe der Datei \textit{index.html} mit dem Befehl
            \texttt{chgrp user index.html} auf die Gruppe, der nur der Nutzer
            \textit{user} angehört. Da \textit{labmate} bereits Besitzer der Datei
            ist, können wir die Rechte einfach mit folgendem Befehl verteilen:
            \texttt{chmod 0750 index.html} .
      \item [14/15.] Mittels cat lässt sich als User nun die Datei \textit{index.html}
            auslesen. Wenn wir jedoch versuchen die Datei mit \textit{nano} zu
            bearbeiten und zu speichern wird uns der Zugriff verweigert, da wir
            keine Schreibrechte als \textit{user} haben.
      \item [16/17.] Der Owner der neuen, kopierten Datei \textit{userindex.html}
            ist jetzt \textit{user}. Also lässt sich diese nun auch als \textit{user}
            modifizieren, da die Zugriffsrechte der Ursprungsdatei beibehalten werden.
      \item [18.] Die Datei \textit{index.html} lässt sich als \textit{user} nicht
            löschen. \textit{user} muss entweder Besitzer der Datei sein oder
            Lese- und Schreibzugriff auf das übergeordnete Verzeichnis besitzen.
    \end{enumerate}

    \subsection{Adnimistration und Aktualisierung}
    \label{sub:Adnimistration und Aktualisierung}

    \begin{enumerate} [1.]
      \item Zuerst sollte man mit dem Befehl \texttt{sudo apt-get update} die
            Indexdateien aller installierten Packages aktualisieren. Auf diese
            Weise ist das System über neue und geupdatete Packages informiert.
            Der Befehl \texttt{sudo apt-get upgrade} aktualisiert, dann alle
            Packages auf die neuste Version. Hierbei werden jedoch keine Packages
            gelöscht. Dies sollte per \texttt{sudo apt-get autoremove} vorgenommen
            werden (siehe \texttt{man apt-get}).
      \item Das Programm \textit{cowsay} gibt mittels \texttt{cowsay [words]} eine
            Kuh mit Sprechblase und den angegebenen Wörtern auf der Konsole zurück.
    \end{enumerate}

    \subsection{Prozesse und Prozessverwaltung}
    \label{sub:Prozesse und Prozessverwaltung}

    \begin{enumerate} [1.]
      \item \texttt{ps} gibt eine Momentaufnahme der per Optionen spezifizierten Prozesse
            auf der Konsole aus. \texttt{top} hingegen gibt eine regelmäßig
            aktualisierende Überischt über alle Prozesse, sowie wie weitere
            Informationen über CPU-Auslastung, Speicherbedarf, etc., auf der Konsole
            aus.
      \item [3.] Beim Ausführen von \texttt{cat /dev/urandom} ist in top der Prozess
            \textit{cat} deutlich zu sehen, da dieser viel Speicher verbraucht.
      \item [4.] Der Prozess \textit{cat} lässt sich nicht als \textit{user} beenden,
            da der Prozess von \textit{labmate} gestartet wurde. Wir benötigen
            Super- oder Rootuserrechte um den Prozess mit \texttt{kill [Prozess-ID]}
            beenden zu können.
      \item [5.] Das System lässt sich nur als sich nur als Superuser neu starten.
            Deshalb benötigen wir den Befehl \texttt{sudo reboot}, um das System als
            \textit{labmate} neuzustarten.
      \item [6.] Mit der Software \texttt{crontab} lassen sich sogenannte \textit{cronjobs}
            also Systemaufgaben anlegen. \textit{cronjobs} bestehen aus einer
            Zeitangabe und den Kommandos, die ausgeführt werden sollen.
            Die Zeitangabe gliedert sich hierbei in Minuten, Stunden, Tag des Monats,
            Monat und Tag der Woche auf. Die Liste aller Cronjobs für einen Nutzer
            lässt sich mittel \texttt{crontab -e} editieren. Um alle 5 Minuten
            einen Zeitstempel an die Datei \texttt{/home/labmate/zeitstempel.txt}
            anzufügen, fügen wir folgenden Befehl zur Liste aller Cronjobs hinzu:
            \texttt{*/5 * * * * date>>/home/labmate/zeitstempel.txt}.
            \texttt{*/5} steht für alle 5 Minuten und der Befehl
            \texttt{date>>/home/labmate/zeitstempel.txt} fügt den aktuellen
            Zeitstempel an die Datei an.
    \end{enumerate}

    \subsection{VMware-Tools installieren}
    \label{sub:VMware-Tools installieren}

    \subsection{Bedienung von VMware}
    \label{sub:Bedienung von VMware}

    \begin{enumerate}[3.]
      \item Beim Pausieren der VM ist zu beobachten, dass das komplette emulierte
            System pausiert wird und damit auch alle Prozesse die in unserer VM
            laufen. Wenn die VM weiterläuft sieht man auch wieder, das \texttt{top}
            sich aktualisiert.
    \end{enumerate}

    \subsubsection{Snapshots}
    \label{subs:Snapshots}

    Snapshots sind Momentaufnahmen der gesamten momentanen VM. Mit diesen Snapshots
    lässt sich der Zustand der VM zum Zeitpunkt der Aufnahme wiederherstellen.
    Im Snapshotmanager sind die Snapshots in chronologischer Reihenfolge angeordnet.

\end{document}
