\documentclass{scrartcl}
\usepackage[utf8]{inputenc}
\usepackage[T1]{fontenc}
\usepackage[ngerman]{babel}
\usepackage[shortlabels]{enumitem}

\author{Patrick Eickhoff, Alexander Timmermann}
\title{Labreport \#1}
\setcounter{section}{1}

\begin{document}
    \maketitle
    \section*{Aufgabe 1}
    \label{sec:Aufgabe 1}

    \subsection{Arbeiten mit der Shell}
    \label{sub:Arbeiten mit der Shell}

    \subsubsection{man ls}
    \label{subs:man ls}
        Mit dem Befehl \texttt{man ls} kann man sich die sog. \textit{man page}
        des ls-Befehls ansehen. Dort werden alle Funktionen des Programmes erläutert
        und ausführlich dokumentiert.

    \subsubsection{ls \--\--help}
    \label{subs:ls --help}
        Mit dem Befehl \texttt{ls \--\--help} kann man sich eine Kurzreferenz des
        ls-Befehls anzeigen lassen. Dort werden die wichtigsten Informationen
        zusammengefasst und auf Hintergrundinfos verzichtet.

    \subsubsection{script}
    \label{subs:script}
        Der Befehl \texttt{script} kann ein Transkript einer Shell-Session
        speichern. Insbesondere für das Schreiben des Lab Reports ist dies
        als Notiz sehr hilfreich.

    \subsection{Benutzerkonten und -verwaltung}
    \label{sub:Benutzerkonten und -verwaltung}

    \begin{enumerate}[1.]
        \item Mit dem Befehl \texttt{adduser labmate} legen wir den User
              \textit{labmate} an. Mit einer interaktiven Abfrage wird das
              Passwort des Users gesetzt.
        \item Mit den Befehlen \texttt{groups} oder \texttt{id} lassen sich die
              Gruppen des Users anzeigen. Zu Beginn befindet sich der User
              lediglich in einer automatisch erstellten Gruppe, die nach dem
              Username benannt ist.
        \item Mit dem Befehl \texttt{addgroup labortests} legen wir die Gruppe
              \textit{labortests} an.
        \item Mit dem Befehl \texttt{usermod -a -G labortests labmate} fügen wir
              den Benutzer \textit{labmate} zur Gruppe \textit{labortests} hinzu.
        \item Damit der neue Benutzer den \textit{sudo}-Befehl benutzen darf,
              muss er zu einer Gruppe hinzugefügt werden, die in der Datei
              \textit{/etc/sudoers} konfiguriert ist. Auf dem Test-System ist
              dies \textit{admin}. Dies bewerkstelligen wir mit dem Befehl
              \texttt{usermod -a -G admin labmate}.
    \end{enumerate}
\end{document}
