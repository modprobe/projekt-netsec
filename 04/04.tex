\documentclass{scrartcl}
\usepackage[utf8]{inputenc}
\usepackage[T1]{fontenc}
\usepackage[ngerman]{babel}
\usepackage[shortlabels]{enumitem}
\usepackage{listings}
\usepackage{xcolor}

\lstdefinestyle{BashInputStyle}{
  language=bash,
  basicstyle=\small\sffamily,
  numbers=left,
  numberstyle=\tiny,
  numbersep=3pt,
  frame=tb,
  columns=fullflexible,
  backgroundcolor=\color{yellow!20},
  linewidth=\linewidth,
  xleftmargin=1mm
}

\author{Patrick Eickhoff, Alexander Timmermann}
\title{Labreport \#4}
\date{}
\setcounter{section}{1}

\begin{document}
    \maketitle
    \section*{Aufgabe 1}
    \label{sec:Aufgabe 1}

    \begin{enumerate}[\textbf{1.}]
        \item[\bf 2.] Die Surfing-VM erhält eine IP-Adresse aus dem Subnetz \texttt{192.168.254.0/24},
                      das Standard-Gateway ist \texttt{192.168.254.1} und der
                      konfigurierte Nameserver ist mit \texttt{192.168.99.1}
                      adressiert.
        \item[\bf 3.] Die Routing-VM besitzt mehrere NICs (\textit{Network Interface Card}).
                      Unter \texttt{eth0} ist das Labor-Netz angebunden, das über das
                      Gateway \texttt{172.16.137.2} angesprochen werden kann.
                      Unter \texttt{eth1} ist das sog. host-only network angeschlossen,
                      über das sich die Surfing-VM verbindet.
        \item[\bf 4.] Bevor die Surfing-VM eine Verbindung herstellen konnte, mussten wir
                      die in der Einleitung aufgeführten Schritte, d.h. Entfernen einiger
                      udev-Regeln, durchführen.
    \end{enumerate}

    \section{Aufgabe 2}
    \label{sec:Aufgabe 2}

    \begin{enumerate}
        \item[\bf 2.] Um den Anforderungen der Aufgabe gerecht zu werden, müssen
                      wir mehrere Optionen von tcpdump kombinieren, sodass wir
                      schlussendlich folgenden Befehl erhalten:
                      \begin{lstlisting}[style=BashInputStyle]
root@routingvm# tcpdump -p -i eth1 -n ``port 53 and host 192.168.254.44''
                      \end{lstlisting}

                      \begin{itemize}
                          \item \texttt{-p} deaktiviert den promiscuous mode, da
                                er von der Virtualisierungsumgebung nicht
                                bereitgestellt werden kann.
                          \item \texttt{-i eth0} beschränkt die Aufzeichnung
                                auf das Interface eth1, über das die Surfing-VM
                                angeschlossen ist
                          \item \texttt{-n} deaktiviert die Auflösung von IP-Adressen
                                zu einem Hostnamen.
                          \item \texttt{port 53} filtert nur nach Paketen, die
                                über Port 53, der DNS zugeordnet ist, gesendet bzw.
                                empfangen werden.
                          \item \texttt{host 192.168.254.44} beschränkt auf Pakete,
                                deren Ziel oder Ausgang mit der IP der Surfing-VM
                                übereinstimmt.
                      \end{itemize}

                      Die Nachrichten, die uns angezeigt werden, protokollieren
                      den DNS-Verkehr, an dem die Surfing-VM beteiligt ist, d.h.
                      sowohl Anfragen als auch Antworten.

                      Das Format der Antworten kann wie folgt beschrieben werden:

                      \texttt{timestamp | source ip | destination ip | dns transaction id | answer records/NS records/additional records | answer}

                      \begin{itemize}
                          \item \texttt{timestamp} ist die Zeit und das Datum, an dem das Paket
                                erfasst wurde
                          \item \texttt{source ip} ist die Quell-IP-Adresse des Pakets
                          \item \texttt{destination ip} ist die Ziel-IP des Pakets
                          \item \texttt{dns transaction id} ist eine eindeutige ID,
                                die Bestandteil des DNS-Protokolls ist und mit der
                                Transaktionen identifiziert werden
                          \item \texttt{answer records} ist die Anzahl an DNS records,
                                die als Antwort zur Verfügung gestellt werden
                          \item \texttt{NS records} ist die Anzahl an DNS records,
                                die die zuständigen Nameserver angeben
                          \item \texttt{additional records} ist die Anzahl an DNS records,
                                die zusätzlich zur Verfügung stehen.
                          \item \texttt{answer} ist dann schließlich die Antwort, die der
                                Nameserver zurückliefert, d.h. entweder eine IP-Adresse
                                (A/AAAA record), ein weiterer Hostname (CNAME record)
                                oder weitere record typen.
                      \end{itemize}

        \item[\bf 3.] Abgewandelt von \textbf{2.} verwenden wir hier folgenden Befehl:
                      \begin{lstlisting}[style=BashInputStyle]
root@routingvm# tcpdump -p -i eth1 -n ``(port 80 or port 443) and src host 192.168.254.44''
                      \end{lstlisting}

                      Wir spezifizieren hier zwei Ports, die ``verodert'' werden,
                      und legen fest, dass die \textbf{Quell}-IP die der Surfing-VM
                      sein muss.

        \item[\bf 4.] Zur Ausgabe der Payload fügen wir den Parameter \texttt{-A}
                      hinzu. Dadurch wird uns der Payload der Pakete in ASCII
                      konvertiert ausgegeben.

                      Bisher werden Pakete jedoch nur verkürzt ausgegeben. Dies
                      rührt daher, dass tcpdump als voreingestellte Paketlänge,
                      die sog. \textit{snap length} 68 bytes verwendet. Wir fügen
                      also als Parameter \texttt{-s 1514} hinzu. Damit schneiden
                      wir die kompletten 1500 bytes des Pakets plus 14 bytes Layer 2
                      Header mit.

        \item[\bf 5.] HTTP Basic Authentication über unverschlüsseltes HTTP ist
                      inhärent unsicher, was wir auch sehen wenn wir den Traffic
                      mitschneiden. Im \texttt{Authorization}-Header werden die
                      Anmeldedaten im Format \texttt{username:password} als base64
                      codiert übertragen und lassen sich sehr leicht mit
                      \begin{lstlisting}[style=BashInputStyle]
root@routingvm# echo -n ``YWxpY2U6c2VocmdlaGVpbQ=='' | base64 -d
                      \end{lstlisting}
                      decodieren.
    \end{enumerate}

    \section{Aufgabe 3}
    \label{sec:Aufgabe 3}

    \begin{enumerate}[\bf 1.]
        \item Zur Demonstration schneiden wir alle HTTP-Interationen mit \texttt{heise.de}
              mit. Die Seite wurde gewählt, da sie nur HTTP unterstützt und HTTPS
              mit \texttt{urlsnarf} nicht mitzuschneiden ist.

              Wir starten \texttt{urlsnarf} mit dem Befehl
              \begin{lstlisting}[style=BashInputStyle]
root@routingvm# urlsnarf -n -i eth1 ``.*heise\.de''
              \end{lstlisting}

              Die Parameter sind dabei weitestgehend deckungsgleich mit denen
              von tcpdump. Der letzte Parameter ist ein regulärer Ausdruck, der
              spezifiziert welche URLs mitgeschnitten werden sollen.

              Der Output erfolgt dabei im \textit{Common Log Format}, wie es z.B.
              auch nginx benutzt.

        \item Mit Hilfe von \texttt{dsniff} können wir uns das Passwort aus \textbf{2.5}
              direkt anzeigen lassen. Dazu starten wir \texttt{dsniff} wie folgt:

              \begin{lstlisting}[style=BashInputStyle]
root@routingvm# dsniff -n -i eth1 'port 80'
              \end{lstlisting}

              Auch hier sind die Parameter wieder deckungsgleich mit tcpdump.
              Rufen wir nun die Testseite auf und authentifizieren uns mit
              \texttt{alice:sehrgeheim}, so zeigt uns dsniff diese als Output an.
    \end{enumerate}

    \section{Aufgabe 4}
    \label{sec:Aufgabe 4}

    \begin{enumerate}[\bf 1.]
        \item[\bf 2.] Ein Capture-Filter spezifiziert, welche Pakete tatsächlich
                      mitgeschnitten werden, während ein Display-Filter alle
                      Pakete mitschneidet, jedoch die Anzeige der Pakete filtert.

        \item[\bf 4.] Schlauerweise schneiden wir nur Traffic mit, der auf dem
                      Interface \texttt{eth1} gesendet bzw. empfangen wird,
                      da über dieses Interface die Surfing-VM angebunden ist.

                      Unabhängig vom Interface können wir auch einen Capture-Filter
                      einrichten, der den Mitschnitt auf die IP-Adresse der
                      Surfing-VM beschränkt.
    \end{enumerate}
\end{document}
