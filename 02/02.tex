\documentclass{scrartcl}
\usepackage[utf8]{inputenc}
\usepackage[T1]{fontenc}
\usepackage[ngerman]{babel}
\usepackage[shortlabels]{enumitem}
\usepackage{listings}

\author{Patrick Eickhoff, Alexander Timmermann}
\title{Labreport \#2}
\date{}
\setcounter{section}{1}

\begin{document}
    \maketitle
    \section*{Aufgabe 1}
    \label{sec:Aufgabe 1}

      \subsection{Zugriff auf \textit{/etc/passwd} und \textit{/etc/shadow}}
      \label{sub:Zugriff auf /etc/passwd und /etc/shadow}

      \begin{enumerate}
        \item[2.] Das Booten des grml-ISO gestaltete sich schwieriger als
                  erwartet. Da das Boot-Menü nur in einem sehr kurzen
                  Zeitfenster zu aktivieren ist, mussten wir in der
                  \textit{.vmx}-Datei von VMWare die Zeile
                  \texttt{bios.forceSetupOnce = True} hinzufügen.
                  Auf diese Weise haben wir einmalig das Öffnen des Bios beim
                  Bootvorgang erzwungen.
        \item[5.] In der Datei \textit{passwd} finden wir Einträge, wie z.B.
                  <Beispiel>. Über die Manualpage (\texttt{man 5 passwd})
                  erfahren wir, das alle Einträge folgende Form besitzen:\\\\
                  \texttt{<login name>:\\
                  <optional encrypted password>:\\
                  <numerical user ID>:\\
                  <numerical group ID>:\\
                  <user name or comment field>:<user home directory>\\
                  {[} : <optional user command interpreter> ] } \\\\
                  Wenn im Passwort Feld ein \textbf{x} steht, finden wir das
                  zugehörige Passwort verschlüsselt in der
                  \textit{shadow}-Datei.
                  Das Format dieser Datei können wir ebenfalls über
                  \texttt{man 5 shadow} ermitteln.  In dieser Datei sind nun
                  tatsächlich die Passwörter der Nutzer angegeben.
          \item[6.] Der einzige Nutzer, der in Mitglied der
                    \textit{admin}-Gruppe ist, heißt georg (admin : georg).
      \end{enumerate}

      \subsection{Auslesen von Kennwörtern}
      \label{sub:Auslesen von Kennwörtern}

      \begin{enumerate}
        \item Ein Passwort zu \textit{hashen} bedeutet, das Passwort mittels
              einer \textit{Hash-Funktion} zu versclüsseln. Eine
              \textit{Hash-Funktion} hat die Eigenschaften, das sie eine
              Einwegfunktion ist, und immer eine Ausgabe fixer Länge erzeugt.
              Zudem sollte sie auch kollisionsfrei sein. Da eine
              \textit{Hash-Funktion} bei gleicher Eingabe immer den gleichen
              \textit{Hash} ausgibt, ist es möglich Anhand einer großen
              Datenbank von \textit{Hashes} die Eingabe zu ermitteln. Um dies
              zu verhindern, werden die Passwörter zusätzlich noch
              \textit{gesaltet}. Dies bedeutet nichts anderes, als das ein
              ein zusätzliches Wort vor oder hinter dem Passwort angehängt
              wird, bevor es \textit{gehasht} wird. Auf diese Weise werden
              für jeden \textit{Salt} unterschiedliche \textit{Hashes} erzeugt.
        \item[4.] Mittels \texttt{sudo john -mode:incremental -users:webadmin
                  -format:crypt shadow} versuchen wir das Passwort des Nutzers
                  \textit{webadmin} per Bruteforce zu entschlüsseln.
                  Nach 5 Minuten brechen wir den Versuch erfolglos ab.
                  Vermutlich ist das Passwort so lang, dass ein Bruteforce-
                  Angriff von enormer Dauer wäre.
        \item[5.] Nachdem wir die Wörterbuch-Dateien mittels \\
                  \texttt{wget http://download.openwall.net/pub/wordlists/
                  all.gz} \\ heruntergeladen und über texttt{gunzip} entpackt
                  haben, können wir unseren Angriff beginnen. Mit dem Befehl: \\
                  \texttt{sudo john -mode:wordlist:/root/all -users:webadmin
                  -format:crypt shadow} \\ starten wir den Wörterbuchangriff und
                  finden heraus, das das Passwort des Nutzers \textit{webadmin}
                  \textbf{mockingbibrd} lautet.
      \end{enumerate}

      \subsection{Setzen von neuen Passwörtern}
      \label{sub:Setzen von neuen Passwörtern}

      \begin{enumerate}
        \item Vermutlich lässt sich das Passwort des Nutzers \textit{georg}
              nicht einfach mit \textit{John the Ripper} ermitteln, da es
              etweder gesaltet ist, zu lang oder einfach nicht in unserer
              Wörterbuch-Datei enthalten ist.
        \item Mittels \texttt{mount -w dev/sda1} mounten wir das Image erneut
              mit Schreibzugriff.
        \item Über den Befel \texttt{chroot . /bin/bash} öffnen wir im
              aktuellen Verzeichnis eine interaktive \textit{bash-Shell}.
        \item Mit \texttt{passwd georg} setzen wir nun das neue Passwort
              \textit{passwort123}.
      \end{enumerate}

    \section{Sichere Speicherung von Kennwörtern}
    \label{sec:Sichere Speicherung von Kennwörtern}

      \subsection{Angriffe mit Hashdatenbanken und Rainbow-Tables}
      \label{sub:Angriffe mit Hashdatenbanken und Rainbow-Tables}

      \begin{enumerate}
        \item Durch den Befehl \texttt{./rcracki} erfahren wir, dass
              \textit{rcrack} mit einem Pfad zu den Rainbowtables, sowie der
              Datei, die entschlüsselt werden soll, ausgeführt wird. Mittels
              Optionen kann angegeben werden, um welche Art von Datei es sich
              handelt.\\\\
              \texttt{./rcrack rainbow\_table\_path [-h hash | -l hashlist |
              -f pwdumpfile]} \\\\
              Nun starten wir unseren Angriff mit folgendem Befehl: \\\\
              \texttt{./rcrack ../md5\_mixalpha-numeric-space\_1-7\_*/*.rti -l
              $\sim$/geheime\_kennwoerter.txt} \\\\
              Wir erhalten folgende 5 Passwörter: ente,ball,borkeni,ulardi,
              avanti. 2 Hashes konnten nicht entschlüsselt werden.
}
        \item Vermutlich wurden die beiden Passwörter, die nicht entschlüsselt
              werden konnten, gesaltet. Die Idee Programm zu schreiben, das
              MD5-Hashes für alle alphanumerischen Passwörter bis zur Länge 7
              berechnet und abspeichert, ist jedoch auch keine passable Lösung.
              Alleine das Berechnen von $$\sum_{i=0}^{7}{62^i} = 3579345993195$$
              Kennwörtern würde enorme Rechenleistung benötigen. Da MD5 immer
              Hashes der Länge 16-Byte generiert würde die Speicherung dieser
              $$3579345993195 * 16B = 5,726953589×10^{13} B
                = 53336 GB
                = 52 TB$$
              an Festplattenspeicher in Anspruch nehmen. Die Rainbowtables aus
              der vorherigen Aufgabe haben hingegen nur ca. 12GB benötigt.
      \end{enumerate}

      \subsection{Eigener Passwort-Cracker}
      \label{sub:Eigener Passwort-Cracker}

        Um das Passwort zu entschlüsseln haben wir uns entschieden einfach
        einen Bruteforce-Angriff durchzuführen. Hierzu lassen wir uns einfach
        mittels for-Schleife und einer Funktion zum erzeugen des karthesischen
        Produkts über einem Alphabet mit einer gegebenen Länge i alle
        alphanumerischen, klein geschriebenen Kennwörter generieren. Jedes
        Kennwort wird dann gesaltet und mittels MD5-Funktion gehasht.
        Die Ausgabe der Funktion vergleichen wir dann mit dem gegebenen Hash.
        Nach einiger Iterationen finden wir tatsächlich heraus, dass das
        Passwort \textbf{s1v3s} lautet. (Für Code siehe \textit{pwcrack.py})

      \subsection{Eigene Kennwort-Speicherfunktion in Java}
      \label{sub:Eigene Kennwort-Speicherfunktion in Java}




\end{document}
