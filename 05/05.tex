\documentclass{scrartcl}
\usepackage[utf8]{inputenc}
\usepackage[T1]{fontenc}
\usepackage[ngerman]{babel}
\usepackage[shortlabels]{enumitem}
\usepackage{listings}
\usepackage{xcolor}

\lstdefinestyle{BashInputStyle}{
  language=bash,
  basicstyle=\small\sffamily,
  numbers=left,
  numberstyle=\tiny,
  numbersep=3pt,
  frame=tb,
  columns=fullflexible,
  backgroundcolor=\color{yellow!20},
  linewidth=\linewidth,
  xleftmargin=1mm
}

\author{Patrick Eickhoff, Alexander Timmermann}
\title{Labreport \#5}
\date{}
\setcounter{section}{1}

\begin{document}
    \maketitle
    \section*{Netzwrkeinstellungen}
    \label{sec:Netzwrkeinstellungen}

    \section{Absichern eines Einzelplatzrechners mit iptables}
    \label{sec:Absichern eines Einzelplatzrechners mit iptables}
    Um das Surfen auf Webseiten zu erlauben, müssen wir den Datenverkehr über
    die Ports 80 (HTTP), 443 (HTTPS) und 53 (DNS) erlauben:
    \begin{lstlisting}[style=BashInputStyle]
      iptables -A OUTPUT -p udp --dport 53 -j ACCEPT
      iptables -A OUTPUT -p tcp --dport 80 -j ACCEPT
      iptables -A OUTPUT -p tcp --dport 443 -j ACCEPT

      iptables -A INPUT -p udp --dport 53 -j ACCEPT
      iptables -A INPUT -p tcp --dport 80 -j ACCEPT
      iptables -A INPUT -p tcp --dport 443 -j ACCEPT
    \end{lstlisting}
    Desweiteren wollen wir sowohl als ICMP-Nachrichten senden und
    empfangen, als auch SSH-Verbindungen (Port 22) aufbauen können:
    \begin{lstlisting}[style=BashInputStyle]
      iptables -A INPUT -p icmp-j ACCEPT
      iptables -A INPUT -p tcp --dport 22 -j ACCEPT

      iptables -A OUTPUT -p icmp -j ACCEPT
      iptables -A OUTPUT -p tcp --sport 22 -j ACCEPT
    \end{lstlisting}
    Letzendlich wollen wir jeglichen anderen Traffic unterbinden:
    \begin{lstlisting}[style=BashInputStyle]
      iptables -A INPUT -j REJECT
      iptables -A OUTPUT -j REJECT
    \end{lstlisting}
\end{document}
