\documentclass[a4paper,10pt]{scrartcl}
\usepackage[utf8]{inputenc}
\usepackage[T1]{fontenc}
\usepackage[ngerman]{babel}
\usepackage[shortlabels]{enumitem}
\usepackage{listings}
\usepackage{xcolor}
\usepackage[title]{appendix}
\usepackage[left=1cm,right=1cm,top=1cm,bottom=3cm]{geometry}
\usepackage{graphicx}
\usepackage{hyperref}

\newcommand*{\Comment}[1]{\hfill\makebox[8.0cm][l]{#1}}%
\lstdefinestyle{BashInputStyle}{
  language=bash,
  basicstyle=\small\sffamily,
  numbers=left,
  numberstyle=\tiny,
  numbersep=6pt,
  frame=tb,
  columns=fullflexible,
  backgroundcolor=\color{yellow!20},
  linewidth=\linewidth,
  xleftmargin=5mm,
  framexleftmargin=5mm,
  escapechar=\&% char to escape out of listings and back to LaTeX
}
\lstset{style=BashInputStyle}

\lstnewenvironment{rootcommands}[1][]{%
  \renewcommand{\thelstnumber}{\#}
  \lstset{style=BashInputStyle,#1}%
}{%
}

\lstnewenvironment{usercommands}[1][]{%
  \renewcommand{\thelstnumber}{\$}
  \lstset{style=BashInputStyle,#1}%
}{%
}

\author{Patrick Eickhoff, Alexander Timmermann}
\title{Labreport \#5}
\date{}

\def\labelitemi{\textbf{--}}

\begin{document}
\maketitle
\section*{Aufgabe 1: Netzwerkeinstellungen}
\label{sec:Netzwerkeinstellungen}

\begin{enumerate}
	\item[\bfseries 2.]
	      \begin{itemize}
	      	\item \underline{Client-VM}
	      	      \begin{description}
	      	      	\item[IP-Adresse:] \texttt{192.168.254.44}
	      	      	\item[Gateway:] \texttt{192.168.254.2}
	      	      	\item[Nameserver:] \texttt{10.1.1.1}
	      	      \end{description}

	      	\item \underline{Router-VM}
	      	      \begin{description}
	      	      	\item[IP-Adresse eth0:] \texttt{172.16.137.222}
	      	      	\item[IP-Adresse eth1:] \texttt{192.168.254.2}
	      	      \end{description}

	      	\item \underline{Server-VM}
	      	      \begin{description}
	      	      	\item[IP-Adresse:] \texttt{172.16.137.144}
	      	      \end{description}
	      \end{itemize}
\end{enumerate}

\section*{Aufgabe 2: Absichern eines Einzelplatzrechners mit iptables}
\label{sec:Absichern eines Einzelplatzrechners mit iptables}

\begin{enumerate}[\bfseries 1.]
	\item Auf der Client-VM sind keine iptables-Regeln vorhanden, die man
	      löschen könnte. Zum Löschen könnte man sonst folgende Befehle benutzen:

	      \begin{rootcommands}
iptables -F                &\Comment{\# flush chains in 'filter' table}&
iptables -t nat -F         &\Comment{\# flush chains in 'nat' table}&
iptables -t mangle -F      &\Comment{\# flush chains in 'mangle' table}&
ipables -X                 &\Comment{\# delete custom chains}&
	      \end{rootcommands}

    	  Mit
    	  \begin{rootcommands}
apt-get update
apt-get install openssh-server
    	  \end{rootcommands}
    	  installieren wir den OpenSSH Server.

	\item Um das Surfen auf Webseiten zu erlauben, müssen wir den Datenverkehr über
	      die Ports 80 (HTTP), 443 (HTTPS) und 53 (DNS) der RouterVM erlauben:
	      \begin{rootcommands}
iptables -A OUTPUT -p udp --dport 53 -j ACCEPT
iptables -A OUTPUT -p tcp --dport 80 -j ACCEPT
iptables -A OUTPUT -p tcp --dport 443 -j ACCEPT
iptables -A INPUT -p udp --dport 53 -j ACCEPT
iptables -A INPUT -p tcp --dport 80 -j ACCEPT
iptables -A INPUT -p tcp --dport 443 -j ACCEPT
    	  \end{rootcommands}
	      Desweiteren wollen wir sowohl als ICMP-Nachrichten senden und
	      empfangen, als auch SSH-Verbindungen (Port 22) aufbauen können:
	      \begin{rootcommands}
iptables -A INPUT -p icmp -j ACCEPT
iptables -A INPUT -p tcp --dport 22 -j ACCEPT
iptables -A INPUT -p udp --dport 53 -j ACCEPT
iptables -A INPUT -p tcp --dport 80 -j ACCEPT
iptables -A INPUT -p tcp --dport 443 -j ACCEPT
        \end{rootcommands}
	Desweiteren wollen wir sowohl als ICMP-Nachrichten senden und
	empfangen, als auch SSH-Verbindungen (Port 22) aufbauen können:
	\begin{rootcommands}
iptables -A INPUT -p icmp-j ACCEPT
iptables -A INPUT -p tcp --dport 22 -j ACCEPT
iptables -A OUTPUT -p icmp -j ACCEPT
iptables -A OUTPUT -p tcp --sport 22 -j ACCEPT
	\end{rootcommands}
	Letzendlich wollen wir jeglichen anderen Traffic unterbinden:
	\begin{rootcommands}
iptables -A INPUT -j REJECT
iptables -A OUTPUT -j REJECT
	\end{rootcommands}

	\item
	      Die SSH-Verbindung von der ClientVM auf die RouterVM
	      (\texttt{ssh user@192.168.254.2}) wird verweigert ("refused"),
	      während die Verbindung von RouterVM auf ClientVM
	      (\texttt{ssh user@192.168.254.1}) problemlos funktioniert.\\
	      Per \texttt{nc -l 5555} setzen wir einen Server auf der CLientVM
	      auf. Wenn wir diesen jedoch von der RouterVM mit
	      \texttt{nc 192.168.254.2 5555} ansprechen wollen, wird die
	      Verbindung verweigert ("refused").\\
	      Wenn wir statt REJECT DROP für unsere Firewall verwenden,
	      bekommen wir bei einem Verbindungsversuch keine Refused-Nachricht
	      mehr zurück, da die Firewall das Packet einfach ohne weitere Aktion
          verwirft, statt eine Abweisungsnachricht zurückzusenden.

	\item Mithilfe dynamischer Regeln können wir einfach definieren, dass ein- und
	      ausgehende Pakete, die zu bereits etablierten Verbindungen gehören
	      (ESTABLISHED,RELATED), automatisch akzeptiert werden:
	      \begin{rootcommands}
iptables -A INPUT -m state --state ESTABLISHED,RELATED -j ACCEPT
iptables -A OUTPUT -m state --state ESTABLISHED,RELATED -j ACCEPT
	      \end{rootcommands}
	      Die restlichen Regeln definieren sich dann wie folgt:
	      \begin{rootcommands}[]
iptables -A OUTPUT -p udp --dport 53 -j ACCEPT
iptables -A OUTPUT -p tcp --dport 80 -j ACCEPT
iptables -A OUTPUT -p tcp --dport 443 -j ACCEPT
iptables -A INPUT -p tcp --dport 22 -j ACCEPT
iptables -A INPUT -p icmp-j ACCEPT
iptables -A INPUT -j REJECT
	      \end{rootcommands}
	      Dynamische Regeln sind sehr angenehm, da sie erlauben Pakete abhängig
	      von ihrem Zustand zu behandeln. So werden deutlich weniger Regeln benötigt,
	      um die Kommunikation bereits aufgebauter Verbindungen zu erlauben.
\end{enumerate}

\section*{Aufgabe 3: Absichern eines Netzwerks}
\label{sec:Absichern eines Netzwerks}

\begin{enumerate}[\bfseries 1.]
	\item
        Folgenden Befehl haben wir in der \textit{rc.local} auf der RouterVM gefunden:\\
        \texttt{iptables -t nat -A POSTROUTING -o eth0 -s 192.168.254.0/24 -j MASQUERADE}\\
        Die \textit{NAT-Tabelle} gibt die Firewall-Regeln für Pakete an, die neue
        Verbindungen aufbauen wollen. Die Postrouting-Chain behandelt Pakete nachdem
        sie geroutet wurden. Der Befehl maskiert, bildet die Source-IP, der Pakete,
        die aus dem internen Netzwerk von Ip-Adressen 192.168.254.0/24 gesendet und
        über das Interface eth0 geroutet wurden, auf eine interne IP-Adresse ab.

	\item
        Die ServerVM von der ClientVM anzupingen funktioniert problemlos,
        da die Pakete über die RouterVM per Masquerading weitergeleitet werden.
        Wenn dann die ICMP-Antwort an der RouterVM ankommt, wird die Destination-IP
        auf die zugehörige IP im internen Netzwerk abgebildet.
        Die ClientVM können wir jedoch nicht von der ServerVM direkt ansprechen, da
        die das Netzwerkinterface der ClientVM nur mit dem Netz unter der RouterVM
        verbunden ist. Da wir das interne Mapping der RouterVM nicht kennen, können
        wir auch nicht die gemappte IP-Adresse der ClientVM ansprechen.

	\item
        Die minimale Firewallkonfiguration sollte wie folgt aussehen:
        \begin{rootcommands}
iptables -A FORWARD -m conntrack --ctstate RELATED,ESTABLISHED -j ACCEPT
iptables -A FORWARD -p tcp -d 172.16.134.144 --dport 80 -j DROP
iptables -A FORWARD -p tcp -d 10.1.1.2 -j DROP
iptables -A FORWARD -p tcp -d 10.0.0.0/8 -j DROP
iptables -A FORWARD -i eth1 -o eth0 -p udp -m udp --dport 53 -j ACCEPT
iptables -A FORWARD -i eth1 -o eth0 -p tcp -m tcp --dport 80 -j ACCEPT
iptables -A FORWARD -i eth1 -o eth0 -p tcp -m tcp --dport 443 -j ACCEPT
iptables -A FORWARD -j DROP
        \end{rootcommands}

	\item
        Damit eine SSH-Verbindung aufgebaut werden kann, müssen wir Forwarding zu Port
        22 erlauben:
        \begin{rootcommands}
iptables -I FORWARD 2 -i eth1 -o eth0 --dport 22 -j ACCEPT
        \end{rootcommands}

	\item
        Zum forwarden der SSH-Verbindung an die ClientVM brauchen wir nur eine Regel:
        \begin{lstlisting}
iptables -t nat -A PREROUTING -i eth0 -p tcp --dport 5022 -j DNAT \
           --to-destination 192.168.254.44:22
		\end{lstlisting}
		Die Pakete die an Port 5022 des äußeren Interfaces eth0 eingehen, werden vorm
		Routing zur ClientVM (192.168.254.44) auf Port 22 weitergeleitet.

	\item
        Nachdem wir eth0 eine weitere IP-Adresse 172.16.137.223 eingerichtet haben,
		geben wir an, dass alle Pakete, die an diese IP-Adresse geschickt werden,
		zur ClientVM weitergeleitet werden:
		\begin{lstlisting}
iptables -t nat -A PREROUTING -i eth0 -d 172.16.137.223 -j DNAT \
         --to-destination 192.168.254.44
		\end{lstlisting}
\end{enumerate}

	\section*{Aufgabe 4: SSH-Tunnel}
	\label{sec:SSH-Tunnel}
	\begin{enumerate}[\bfseries 1.]
		\item
		    Als erstes konfigurieren wir die Firewall:
            \begin{rootcommands}
iptables -A FORWARD -m contrack --ctstate ESTABLISHED,RELATED -j ACCEPT
iptables -A FORWARD -i eth1 -o eth0 -p udp --dport 53 -j ACCEPT
iptables -A FORWARD -i eth1 -o eth0 -p tcp --dport 22 -j ACCEPT
iptables -A FORWARD -j DROP
            \end{rootcommands}

	\item
		Ein SSH-Tunnel lässt sich einfach über den ssh-Befehl aufbauen\footnote{Quelle: \url{http://blog.trackets.com/2014/05/17/ssh-tunnel-local-and-remote-port-forwarding-explained-with-examples.html}}:
        \begin{usercommands}
ssh -L 9000:localhost:80 user@172.16.137.144
        \end{usercommands}
		Mit obigem Befehl binden wir Port 80 der ServerVM an Port 9000 der ClientVM
		per SSH-Verbindung. Dann lässt sich über \textit{localhost:9000} auf den
		Webserver der ServerVM zugreifen. Über Wireshark sehen wir nur die
		SSH-Pakete.\\

	\item
	    Local Forwarding zum Browsen wäre sehr aufwendig, da jeder Port, den
		wir zum Surfen brauchen, einzeln geforwarded werden muss.
		Statdessen benutzen wir Dynamic Forwarding, um unsere SSH-Verbindung als
		SOCKS-Proxy zu verwenden\footnote{Quelle: \url{https://help.ubuntu.com/community/SSH/OpenSSH/PortForwarding}}:
        \begin{usercommands}
ssh -D 5555 user@172.16.137.144
        \end{usercommands}
		Im Browser müssen wir nun den Proxy konfigurieren:\\
        \includegraphics[width=0.4\textwidth]{proxy.png}

    \item
		Zuerst starten wir auf der ClientVM einen Server mittels Netcat
		\texttt{nc -L 5555} und bauen dann unseren SHH-Tunnel zum Remote-Forwarding
		auf
        \begin{usercommands}
ssh -R 9000:localhost:5555 user@172.16.137.144
        \end{usercommands}
        Nun können wir uns auf der ServerVM mittels \texttt{nc localhost 9000} verbinden.
	\end{enumerate}

	\section*{Aufgabe 5: OpenVPN}
	\label{sec:OpenVPN}
	\begin{enumerate}[\bfseries 1.]
		\item
            Filterregeln:
		    \begin{rootcommands}
iptables -A FORWARD -m conntrack --ctstate ESTABLISHED,RELATED -j ACCEPT
iptables -A FORWARD -i eth1 -o eth0 -j ACCEPT
iptables -A FORWARD -i eth0 -o eth1 -j DROP
		    \end{rootcommands}

		\item
		    Der Server und Client sind wie in der \textit{client.conf} (App. A)
		    und \textit{server.conf} (App. B) angegeben, konfiguriert.
		\item
            Die Firewall erlaubt Pakete die Port 1194 für OpenVPN
		    ansteuern und Pakete die vom Interface tun0, welches von der VPN
		    genutzt wird, eingehen.
		    Ausgehen dürfen nur Pakete die zu einer bereits etablierten
		    VPN-Verbindung gehören:
		    \begin{rootcommands}
iptables -A INPUT -i tun0 -j ACCEPT
iptables -A INPUT -p tcp --dport 1194 -j ACCEPT
iptables -A INPUT -j DROP
iptables -A OUTPUT -m conntrack --ctstate ESTABLISHED,RELATED -j ACCEPT
iptables -A OUTPUT -j DROP
            \end{rootcommands}
		\item
		      Wir starten erst VPN-Client und -Server:
		      \texttt{openvpn client.conf} und \texttt{openvpn server.conf}.
		      Auf der ClientVM sehen wir dann bei erfolgreicher Verbindung:\\
              \begin{verbatim}
Peer Connection Initiated with [AF_INET]172.16.137.222:54286
Initialization Sequence Completed
              \end{verbatim}
		      Auf der ServerVM respektive:\\
              \begin{verbatim}
Peer Connection Initiated with [AF_INET]172.16.137.144:1194
Initialization Sequence Completed
              \end{verbatim}
		\item
		      Für den Tunnel der VPN wird das Interface tun0 verwendet.\\
		      Der Webserver der ServerVM lässt sich einfach im Browser über
		      \textit{10.8.0.1:80} aufrufen.\\
		      Der Aufbau einer SSH-Verbindung ist möglich, so lange er über das
              VPN erfolgt. Der Client kann in unserer Konfiguration unter
              \texttt{10.8.0.2} erreicht werden, also:
              \begin{usercommands}
ssh user@10.8.0.2
              \end{usercommands}
	\end{enumerate}

\section*{Aufgabe 6: HTTP-Tunnel}
\label{sec:Aufgabe 6: HTTP-Tunnel}
\begin{enumerate}[\bfseries 1.]
	\item
        Firewall:
        \begin{rootcommands}
iptables -A FORWARD -i eth1 -o eth0 -p tcp --dport 80 -j ACCEPT
iptables -A FORWARD -i eth1 -o eth0 -p udp --dport 53 -j ACCEPT
		\end{rootcommands}

	\item
        Es gibt mehrere Wege, die Firewall zu überlisten.\\
        Ein Weg ist es, Pakete die bei der ServerVM auf Port 80 eingehen,
        an Port 22 Weiterleiten:
		\begin{rootcommands}
iptables -t nat -A PREROUTING -p tcp --dport 80 -j REDIRECT --to-ports 22
		\end{rootcommands}

        Eine andere Möglichkeit ist, sshd auf der ServerVM auf Port 80 lauschen
        zu lassen, indem man in \texttt{/etc/ssh/sshd\_config} die Definition
        \texttt{Port 22} durch \texttt{Port 80} ersetzt. Nun muss man sshd neustarten
        und Apache, was bisher Port 80 belegt, stoppen:

        \begin{rootcommands}
service apache2 stop
service sshd restart
        \end{rootcommands}

	\item
        Firewall:
	    \begin{rootcommands}
iptables -A INPUT -i eth1 -p tcp --dport 3128 -j ACCEPT
iptables -A INPUT -i eth1 -j DROP
iptables -A FORWARD -j DROP
iptables -A OUTPUT -o eth0 -p udp --dport 53 -j ACCEPT
iptables -A OUTPUT -o eth0 -p tcp --dport 80 -j ACCEPT
iptables -A OUTPUT -o eth0 -p tcp --dport 443 -j ACCEPT
iptables -A OUTPUT -o eth0 -j DROP
	    \end{rootcommands}

	\item
        In Firefox müssen wir als HTTP-Proxy nur \textit{192.168.254.2:3128}
        einstellen.

	\item
        Als erstes verbinden wir uns über Netcat mit dem SSH-Server der ServerVM:
        \begin{usercommands}
nc -x192.168.254.2:3128 -Xconnect 172.16.134.144 80
        \end{usercommands}
		Danach konfigurieren wir unseren ssh-Befehl mittels Corkscrew in der
		\textit{~/.ssh/config}:\\
		\begin{lstlisting}
Host *
  ProxyCommand corkscrew 192.168.254.2 3128 %h %p
		\end{lstlisting}
		\item Wir überlisten die Firewall einfach, indem wir Packete die bei der
		      ServerVM auf Port 80 eingehen an Port 22 Weiterleiten:
		      \begin{lstlisting}
  iptables -t nat -A PREROUTING -p tcp --dport 80 -j REDIRECT --to-ports 22
		\end{lstlisting}
		\item Firewall:
		      \begin{lstlisting}
  iptables -A INPUT -i eth1 -p tcp --dport 3128 -j ACCEPT
  iptables -A INPUT -i eth1 -j DROP
  iptables -A FORWARD -j DROP
  iptables -A OUTPUT -o eth0 -p udp --dport 53 -j ACCEPT
  iptables -A OUTPUT -o eth0 -p tcp --dport 80 -j ACCEPT
  iptables -A OUTPUT -o eth0 -p tcp --dport 443 -j ACCEPT
  iptables -A OUTPUT -o eth0 -j DROP
		\end{lstlisting}
		\item In Firefox müssen wir als HTTP-Proxy nur \textit{192.168.254.2
			3128} einstellen.
			\item
			      Als erstes verbinden wir uns über Netcat mit dem SSH-Server der ServerVM:\\
			      \texttt{nc -x192.168.254.2:3128 -Xconnect 172.16.134.144 80}\\
			      Danach konfigurieren wir unseren ssh-Befehl mittels Corkscrew in der
			      \textit{~/.ssh/config}:\\
			      \begin{lstlisting}
 Host *
  ProxyCommand corkscrew 192.168.254.2 3128 %h %p
			\end{lstlisting}
			Jetzt können wir einfach einen HTTP-Tunnel via SSH aufbauen:\\
			\texttt{ssh -p 80 user@172.16.134.144}
			\item Nachdem wir httptunnel installiert haben, müssen wir nur mit
			      \texttt{hts -F localhost 80} den Server auf der ServerVM öffnen und
			      den Client mittels \texttt{htc -P 192.168.254.2:3128 -F 8888
			      172.16.134.144 80} auf der ClientVM starten. Danach können wir einen
			      HTTP-Tunnel von der ClientVM zur ServerVM aufbauen:
			      \texttt{ssh -p 8888 user@localhost}
		\end{enumerate}

\pagebreak

\begin{appendices}
    \section{Server-Config}
    \lstinputlisting[caption=server.conf]{server.conf}
    \section{Client-Config}
    \lstinputlisting[caption=client.conf]{client.conf}
\end{appendices}
\end{document}
