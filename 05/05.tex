\documentclass{scrartcl}
\usepackage[utf8]{inputenc}
\usepackage[T1]{fontenc}
\usepackage[ngerman]{babel}
\usepackage[shortlabels]{enumitem}
\usepackage{listings}
\usepackage{xcolor}

\lstdefinestyle{BashInputStyle}{
  language=bash,
  basicstyle=\small\sffamily,
  numbers=left,
  numberstyle=\tiny,
  numbersep=3pt,
  frame=tb,
  columns=fullflexible,
  backgroundcolor=\color{yellow!20},
  linewidth=\linewidth,
  xleftmargin=1mm
}
\lstset{style=BashInputStyle}

\author{Patrick Eickhoff, Alexander Timmermann}
\title{Labreport \#5}
\date{}

\begin{document}
    \maketitle
    \section{Netzwerkeinstellungen}
    \label{sec:Netzwerkeinstellungen}

    \section{Absichern eines Einzelplatzrechners mit iptables}
    \label{sec:Absichern eines Einzelplatzrechners mit iptables}

    \setcounter{subsection}{1}
    \subsection{}
    \label{sub:2.2}

    Um das Surfen auf Webseiten zu erlauben, müssen wir den Datenverkehr über
    die Ports 80 (HTTP), 443 (HTTPS) und 53 (DNS) der RouterVM erlauben:
    \begin{lstlisting}
      iptables -A OUTPUT -p udp --dport 53 -j ACCEPT
      iptables -A OUTPUT -p tcp --dport 80 -j ACCEPT
      iptables -A OUTPUT -p tcp --dport 443 -j ACCEPT

      iptables -A INPUT -p udp --dport 53 -j ACCEPT
      iptables -A INPUT -p tcp --dport 80 -j ACCEPT
      iptables -A INPUT -p tcp --dport 443 -j ACCEPT
    \end{lstlisting}
    Desweiteren wollen wir sowohl als ICMP-Nachrichten senden und
    empfangen, als auch SSH-Verbindungen (Port 22) aufbauen können:
    \begin{lstlisting}
      iptables -A INPUT -p icmp-j ACCEPT
      iptables -A INPUT -p tcp --dport 22 -j ACCEPT

      iptables -A OUTPUT -p icmp -j ACCEPT
      iptables -A OUTPUT -p tcp --sport 22 -j ACCEPT
    \end{lstlisting}
    Letzendlich wollen wir jeglichen anderen Traffic unterbinden:
    \begin{lstlisting}
      iptables -A INPUT -j REJECT
      iptables -A OUTPUT -j REJECT
    \end{lstlisting}

    \subsection{}
    \label{sub:2.3}
    \begin{itemize}
      \item
      Die SSH-Verbidung von der CLientVM auf die RouterVM
      (\texttt{ssh user@192.168.254.2}) wird verweigert ("refused"),
      während die Verbindung von RouterVM auf ClientVM
      (\texttt{ssh user@192.168.254.1}) problemlos funktioniert.
      \item
      Per \texttt{nc -l 5555} setzen wir einen Server auf der CLientVM
      auf. Wenn wir diesen jedoch von der RouterVm mit
      \texttt{nc 192.168.254.2 5555} ansprechen wollen, wird die
      Verbindung verweigert ("refused").
      \item
      Wenn wir statt REJECT DROP für unsere Firewall verwenden,
      bekommen wir bei einem Verbindungsversuch keine Refused-Nachricht
      mehr zurück. Da die Firewall das Packet einfach ignoriert.
    \end{itemize}

  \subsection{}
  \label{sub:2.4}
  Mithilfe dynaischer Regeln können wir einfach definieren, dass ein- und
  ausgehende Packete, die zu bereits etablierten Verbindungen gehören
  (ESTABLISHED,RELATED), automatisch akzeptiert werden:
  \begin{lstlisting}
    iptables -A INPUT -m state --state ESTABLISHED,RELATED -j ACCEPT
    iptables -A OUTPUT -m state --state ESTABLISHED,RELATED -j ACCEPT
  \end{lstlisting}
  Die restlichen Regeln definieren sich dann wie folgt:
  \begin{lstlisting}[]
    iptables -A OUTPUT -p udp --dport 53 -j ACCEPT
    iptables -A OUTPUT -p tcp --dport 80 -j ACCEPT
    iptables -A OUTPUT -p tcp --dport 443 -j ACCEPT

    iptables -A INPUT -p tcp --dport 22 -j ACCEPT
    iptables -A INPUT -p icmp-j ACCEPT
    iptables -A INPUT -j REJECT
  \end{lstlisting}
  Dynamische Regeln sind sehr angenehm, da sie erlauben Packete abhängig
  von ihrem Zustand zu behandeln. So werden deutlich weniger Regeln benötigt,
  um die Kommunikation bereits aufgebauter Verbindungen zu erlauben.
\end{document}
